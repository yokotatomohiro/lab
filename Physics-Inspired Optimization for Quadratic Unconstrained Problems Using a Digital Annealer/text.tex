要約

富士通のデージタルアニーラはフルコネクトでQUBO問題を解くために設計されている.
現在1024変数までの問題を解く. 
DAはSAを元にしているが効率的な並列試行方式と動的エスケープメカニズムを利用している. 

実験の結果
SAと並列焼き戻しシングルコアの実装に対してDAは, 現在のバイモーダルまたはガウスカップリングによる完全結合スピングラス問題に対して, 約二桁の時間短縮を実現した. 
だが, 疎な二次元スピングラス問題に対してはスピードアップを示さない. 
また, パラレルテンパリングデジタルアニーラの早期実装のベンチマークも実施した. 
その結果, 二峰性障害を伴う平均困難性の完全に関連した問題に対して, 他のアルゴリズムよりも良い結果を示唆した. 

1. Introduction 要約
量子アニーラの利点を探す努力が続けられているが, まだ見つかっていない. 
ベンチマークに対してSAのような古典的なものと量子アニーラとで比較すると量子アニーラによってアップが見られていたが, その後, より強力な古典的アルゴリズムが考えれて効率的に解けるようになった.

富士通ラボラトリーズはDAとして知られる完全接続されたQUBO問題を解くCMOSハードウェアを開発した. 
DAはSAを参考にしているが大規模な並列が利用できる点で異なる. 
DAに加えて, PTDAと呼ぶアルゴリズムエンジンに並列焼き戻しモンテカルロを使用するDAのバージョンが利用可能である. 
SAや焼き戻しモンテカルロなどの最適化技術はD-waveなどの量子ハードの性能を凌駕している. 
本稿では, スピングラス問題についてDAとPTDAをベンチマークし, 等エネルギークラスタ移動を用いたSAおよび並列焼き戻しモンテカルロとを比較する. 

第2章ではベンチマークで利用したアルゴリズムについての説明
第3章では並列焼き戻しの利点を検証する. 
第4章では計算時間を計測するために使用する方法についての説明. 
第5章ではベンチマークされた問題を紹介する. 
第6章では実験結果を提示し議論する. 
第7章ではその結果を示す. 

2. Algorithms 利用するアルゴリズム

B. デジタルアニーラのアルゴリズム
DAのアルゴリズムはSAに基づいているが以下の3点で異なる. 
・任意の状態から全ての実行を開始する
・各モンテカルロステップが全ての変数の反転を並列して考慮する並列試行方式を使用する
・DAは動的オフセットと呼ばれるエスケープメカニズムを採用している

C. Parallel Tempering with Isoenergetic Cluster Moves 等エネルギクラスタ移動による並列焼き戻し
PTは温度空間内をランダムに移動し, 一時的に高温に移動することでエネルギー障壁を克服できる. 
ICMをPTに追加することで一度に複数の変数を反転させることができる. 

D. The Parallel Tempering Digital Annealer's Algorithm
DAのアルゴリズムはSAに基づいており, PTがSAよりも優れた結果をもたらすことが多いため(たとえば, 参考文献[30]を参照), Fufitsu Laboratories は Parallel Tempering Digital Annealer(PTDA)も開発した. 
私たちはPTDAの早期実施へのアクセスを持っていた. 
PTDAでは, パラレルトライアル方式, パラレルアップデート, 動的オフセットメカニズムの使用など, 各レプリカのスイープはDAと同様に実行されるが, PT移動はCPU上で実行される. 

