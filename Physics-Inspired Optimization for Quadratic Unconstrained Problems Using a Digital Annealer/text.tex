要約

富士通のデージタルアニーラはフルコネクトでQUBO問題を解くために設計されている.
現在1024変数までの問題を解く. 
DAはSAを元にしているが効率的な並列試行方式と動的エスケープメカニズムを利用している. 

実験の結果
SAと並列焼き戻しシングルコアの実装に対してDAは, 現在のバイモーダルまたはガウスカップリングによる完全結合スピングラス問題に対して, 約二桁の時間短縮を実現した. 
だが, 疎な二次元スピングラス問題に対してはスピードアップを示さない. 
また, パラレルテンパリングデジタルアニーラの早期実装のベンチマークも実施した. 
その結果, 二峰性障害を伴う平均困難性の完全に関連した問題に対して, 他のアルゴリズムよりも良い結果を示唆した. 

1. Introduction 要約
量子アニーラの利点を探す努力が続けられているが, まだ見つかっていない. 
ベンチマークに対してSAのような古典的なものと量子アニーラとで比較すると量子アニーラによってアップが見られていたが, その後, より強力な古典的アルゴリズムが考えれて効率的に解けるようになった.

富士通ラボラトリーズはDAとして知られる完全接続されたQUBO問題を解くCMOSハードウェアを開発した. 
DAはSAを参考にしているが大規模な並列が利用できる点で異なる. 
DAに加えて, PTDAと呼ぶアルゴリズムエンジンに並列焼き戻しモンテカルロを使用するDAのバージョンが利用可能である. 
SAや焼き戻しモンテカルロなどの最適化技術はD-waveなどの量子ハードの性能を凌駕している. 
本稿では, スピングラス問題についてDAとPTDAをベンチマークし, 等エネルギークラスタ移動を用いたSAおよび並列焼き戻しモンテカルロとを比較する. 

第2章ではベンチマークで利用したアルゴリズムについての説明
第3章では並列焼き戻しの利点を検証する. 
第4章では計算時間を計測するために使用する方法についての説明. 
第5章ではベンチマークされた問題を紹介する. 
第6章では実験結果を提示し議論する. 
第7章ではその結果を示す. 

2. Algorithms 利用するアルゴリズム

B. デジタルアニーラのアルゴリズム
DAのアルゴリズムはSAに基づいているが以下の3点で異なる. 
・任意の状態から全ての実行を開始する
・各モンテカルロステップが全ての変数の反転を並列して考慮する並列試行方式を使用する
・DAは動的オフセットと呼ばれるエスケープメカニズムを採用している

C. Parallel Tempering with Isoenergetic Cluster Moves 等エネルギクラスタ移動による並列焼き戻し
PTは温度空間内をランダムに移動し, 一時的に高温に移動することでエネルギー障壁を克服できる. 
ICMをPTに追加することで一度に複数の変数を反転させることができる. 

D. The Parallel Tempering Digital Annealer's Algorithm
DAのアルゴリズムはSAに基づいており, PTがSAよりも優れた結果をもたらすことが多いため(たとえば, 参考文献[30]を参照), Fufitsu Laboratories は Parallel Tempering Digital Annealer(PTDA)も開発した. 
私たちはPTDAの早期実施へのアクセスを持っていた. 
PTDAでは, パラレルトライアル方式, パラレルアップデート, 動的オフセットメカニズムの使用など, 各レプリカのスイープはDAと同様に実行されるが, PT移動はCPU上で実行される. 


3. Parallel-trial Versus Single-trial Monte Carlo
システムの温度による一回の試行ごとの単一試行と並列試行の許容確率を測定する. 
その時の結果が図1である. 
また, 並列試行の優位性を考えるために, 図2のように並列試行の許容確率を単一試行の許容確率で割った値を示した. 

4. Scaling Analysis
ベンチマークの目的は, 問題サイズが増加するにつれて, 計算量がどの程度拡大するかを定量化することである. 
一般的アプローチは0.999の確率で少なくとも一回, 参照コストを見つけるために必要な合計時間を測定するので, この時間をTTSで計算する. 

TTSを導出する手順はDAとPTDAとで異なる. 

5. Benchmarking Problems
ベンチマークで使用する問題について. 

二次元-二峰性
  二峰性分布に従って結合が選択されるトーラス(周期的境界)上の二次元スピングラス問題. すなわち, それらは等しい確率で{-1,1}から値をとる. 
  
二次元-ガウス
  結合が10^5でスケールされた, 平均0で標準偏差が1のガウス分布から選択される二次元スピングラス問題. 
  
SK-二峰性-完全グラフ上のスピングラス問題 - 別名Sherrington-Krikpatrick(SK)スピングラス問題
  カップリングは二峰性分布に従って選択される. つまり, それらは等しい確率で{-1,1}から値をとる. 
  
SK-ガウス
  結合が平均0標準偏差1のガウス分布から選択されるSKスピングラス問題でありスケールは10^5

SK問題は二次元問題よりも困難であり, 15分の制限時間内に最適な解を見つけられなかったので, 64個の変数をもつSK問題について, それぞれ最適解を見つけるためにBiq Mac Solver と Biq Crunch による半正定値分岐限定法を使用した. 

6. Results and Discussion
本稿の全てのTTSプロットでのエラーバーはTTSプンプの平均と5番目と95番目の百分位数を表している. 

A. 二次元スピングラス問題を解いた時の結果
PT+ICMはどの問題に対しても低いTTSを持ち, 明確な利点がある. 
SAは小規模の問題ではTTSが低くなるが, サイズが大きくなるとDAとSAは同等のTTSを示す. 
しかし, DAは現在の精度限界でもSAよりも優れている. 
反復回数を10^7に増やしたが, サイズが400以上の二次元ガウス問題の80%以上を解くことができなかった

二次元二峰性問題について
図4aでは, 与えられた問題のサイズとスイープ数に対して, DAはSAよりも高い成功確率に達している. 
問題サイズが大きくなるにつれてDAとSAの平均成功曲線の差はそれほど顕著にならない. 
図4bは, サイズ0124の100個の問題インスタンスのうち52個の問題インスタンスに対してより高い成功確率をもたらした. 

二次元二峰性問題に対しては, 


