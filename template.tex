%%「論文」,「レター」,「レター(C分冊)」,「技術研究報告」などのテンプレート
%% v3.2 [2019/03/19]
%% 1. 「論文」
\documentclass[paper]{ieicej}
%\usepackage[dvips]{graphicx}
%\usepackage[dvipdfmx]{graphicx,xcolor}
\usepackage[T1]{fontenc}
\usepackage{lmodern}
\usepackage{textcomp}
\usepackage{latexsym}
%\usepackage[fleqn]{amsmath}
%\usepackage{amssymb}

\setcounter{page}{1}

\field{}
\jtitle{イジング計算機での利用に向けた$l1$ノルムのQUBO形式について}
\etitle{}
\authorlist{%
 \authorentry{aaa}{Tomohiro Yokota}{}\MembershipNumber{}
 %\authorentry{和文著者名}{英文著者名}{所属ラベル}\MembershipNumber{}
 %\authorentry[メールアドレス]{和文著者名}{英文著者名}{所属ラベル}\MembershipNumber{}
 %\authorentry{和文著者名}{英文著者名}{所属ラベル}[現在の所属ラベル]\MembershipNumber{}
}
\affiliate[]{}{}
%\affiliate[所属ラベル]{和文所属}{英文所属}
%\paffiliate[]{}
%\paffiliate[現在の所属ラベル]{和文所属}

\begin{document}
\begin{abstract}
  本稿では量子アニーリングを含むイジングモデルを用いたアニーリング法においてスパース推定を可能にするために、近年ReLUタイプ関数のQUBO形式を導出するために用いられたLegendre変換とWolfeの双対定理を利用した。さらに$l1$ノルムに対してこれらの変換を素朴に適用した場合、余分な変数も現れることが明らかになった。最終的に余分な変数を取り除くことで、より簡略化されたQUBO形式を導出する。
\end{abstract}
\begin{keyword}
%和文キーワード 4〜5語
\end{keyword}
\begin{eabstract}
%英文アブストラクト 100 words
\end{eabstract}
\begin{ekeyword}
%英文キーワード
\end{ekeyword}
\maketitle

\section{はじめに} 
近年、最適化問題の近似解を得ることに特化したアニーリングマシンが開発・提供されており、代表的なものにカナダのD-Wave.Incの''D-Wave 2000"や富士通の''Digital Annealer"、日立の''CMOSアニーリングマシン"などがある。最適化問題は、データマイニングや機械学習などの様々な分野で利用されている。特に、量子アニーリング法は門脇と西森によって提案され、同様の考え方をもつ断熱量子計算の考え方が注目を集めた。近年では実用化のための研究が行われており、
アニーリングマシンのハードウェアはイジングモデルのハミルトニアンを基にしているので、最適化問題のコスト関数をQUBO形式に変換する必要がある。連続変数については二進数展開を行うことでイジングタイプの変数に変換することができるが、コスト関数を直接QUBO形式に再定式化する系統的な導出方法は示されていない。しかし、論理ゲートを含むいくつかのものに対してはLucasによって再定式化がされている。近年では、Legendre変換を用いた$q$-loss関数のQUBO形式での導出が行われた.また,近年発表されたReLUタイプ関数のQUBO形式での導出ではLegendre変換だけでは不十分であり,新たにWolfeの双対定理を用いて導出を行っていた.

本稿で導出を行うl1ノルムは$q$-loss関数やReLUタイプ関数のようにコスト関数のペナルティとして用いることで解に''スパース性"を与えることができる.スパース推定は画像処理や機械学習の分野で広く利用されており,データ数の削減や高次元で複雑なデータから関連のあるデータのみを抽出できることからとても重要である.

本稿の構成は次のようになる。まず第二章ではl1ノルムについての説明と利用例についての説明を行う。第三章ではQUBO形式とイジングモデルについての説明を行う。第四章では先行研究で行われた$q$-loss関数のQUBO形式への変換と導出に用いられたLegendre変換についての説明を行う。第五章ではReLUタイプ関数のQUBO形式を導出する際に追加で用いられたWolfeの双対定理についての説明とl1ノルムに対してLegendre変換とWolfeの双対定理を素朴に適用しることでQUBO形式への変換を行う。第六章では、第五章で導出された数式を見直すことで、余分な変数の削減を行う。第七章では、まとめと今後の展望について述べる。


\section{l1ノルムとスパース推定}
\section{QUBO形式とイジングモデル}
一章で説明したように,アニーリングマシンはイジングモデルという物理モデルがハードウェアとして実装されており,アニーリングを高速で実行することで近似最適解を出力する.イジングモデルは次のエネルギー関数で表現される.
\begin{eqnarray}
 H = \sum_{i\neq j}{J_{ij}\sigma_{i}\sigma_{j}}+\sum_{i}{h_{i}\sigma_{i}}, \label{Ising_model}
\end{eqnarray}
ここで,$\sigma_{i}$は$\sigma\in\{-1,+1\}$の入力変数であり,$J_{i,j}$は二体相互作用のパラメータ,$h_{i}$は一体相互作用のパラメータである.アニーリングマシンに相互作用のパラメータ$J_{i,j},h_{i}$を与えることでエネルギーが最小となる$\sigma$の組み合わせ出力する
\subsection{QUBO形式}
最適化問題を二値変数$\sgima\in\{-1,+1\}$を用いたイジングモデルに直接変換することは難しい.そこで,$\sigma$の代わりに$q\in\{0,1\}$を用いた形式にする.これはQUBO(Quadratic Unconstrained Binary Optimization)形式と呼ばれ、エネルギー関数は次のように表される.
\begin{eqnarray}
 H = \sum_{i\neq j}{\widetilde{J}_{i,j}q_{i}q_{j}}+\sum_{i}{\widetilde{h}_{i}q_{i}}, \label{QUBO_model}
\end{eqnarray}

\section{先行研究:$q$-loss関数のQUBO形式での導出}
この章では先行研究で行われた$q$-loss関数のQUBO形式での導出について紹介する。$q$-loss関数は式と定義され、グラフの外形は図のようになる。
\begin{eqnarray}
 L_{q}(m) = \min{[(1-q)^{2},(\max{[0,1-m]})^{2}]},
\end{eqnarray}
ここで、$q\in(-\infty,0]$のパラメータであり、$m$は実変数である。

分類問題において、この$q$-loss関数をコスト関数のペナルティ項と用いることで、識別境界から大きく離れたラベルノイズが存在してもロバスト性を確保するすることができる。

式は$\min$関数と$\max$関数が含まれているため二次形式ではなく、またQUBO形式でもない。そこで、Legendre変換を用いることで式をQUBO形式に変換する。次の節ではLegendre変換についての説明を行う。
\subsection{Legendre変換}
この節では、Legendre変換についての簡単な説明を行う。$f_{L}(x)$を凸関数とするとき、次のような関数$f^{*}_{L}(\eta)$を考える。
\begin{eqnarray}
 f^{*}_{L}(\eta) &=& \min{\{\eta x-f(x)\}} \\
                 &=& -\max{\{f(x)-\eta x\}}
\end{eqnarray}
$\eta$は追加変数であり、$x$に関する関数から$\eta$に関する関数へと変換している。ここでこのような$f(x)_{L}$から$f^{*}_{L}(\eta)$への変換をLegendre変換と呼び、$f^{*}_{L}(\eta)$のことを共役関数と呼ぶ。また、元の関数$f(x)_{L}$が凸関数の場合、共役関数の共役関数つまり、$f^{*}_{L}(\eta)$の共役関数$f^{**}_{L}$は元の関数と一致するという性質を持つ。
\subsection{$q$-loss関数へのLegendre変換の適用}
先行研究では、式を凸関数の形に変換した後にLegendre変換と式変形を行うことで、次のようなQUBO形式への変換を行なった。
\begin{eqnarray}
 L_{q}(m) = \min_{t}{\left\{(m-t)^{2}+(1-q)^{2}\frac{1-\mathrm{sign}(t-1)}{2}\right\}}
\end{eqnarray}
ここで、変数$t$はLegendre変換を適用することで追加された変数である。また、$m,t$は連続変数であるが、二進数展開を用いることで二体相互作用で表現可能であり、sign関数についても一体相互作用で表現することができる。

\section{l1ノルムの素朴なQUBO形式での導出}
l1
\subsection{Wolfeの双対定理}
\section{余分な変数の削除}


\ack %% 謝辞

%\bibliographystyle{sieicej}
%\bibliography{myrefs}
\begin{thebibliography}{99}% 文献数が10未満の時 {9}
\bibitem{}
\end{thebibliography}

\appendix
\section{}

\begin{biography}
\profile*{aaa}{aaa}{aaa}
%\profile{会員種別}{名前}{紹介文}% 顔写真あり
%\profile*{会員種別}{名前}{紹介文}% 顔写真なし
\end{biography}

\end{document}
