%%「論文」,「レター」,「レター(C分冊)」,「技術研究報告」などのテンプレート
%% v3.2 [2019/03/19]
%% 1. 「論文」
\documentclass[paper]{ieicej}
%\usepackage[dvips]{graphicx}
%\usepackage[dvipdfmx]{graphicx,xcolor}
\usepackage[T1]{fontenc}
\usepackage{lmodern}
\usepackage{textcomp}
\usepackage{latexsym}
%\usepackage[fleqn]{amsmath}
%\usepackage{amssymb}

\setcounter{page}{1}

\field{}
\jtitle{イジング計算機での利用に向けた$l1$ノルムのQUBO形式について}
\etitle{}
\authorlist{%
 \authorentry{aaa}{Tomohiro Yokota}{}\MembershipNumber{}
 %\authorentry{和文著者名}{英文著者名}{所属ラベル}\MembershipNumber{}
 %\authorentry[メールアドレス]{和文著者名}{英文著者名}{所属ラベル}\MembershipNumber{}
 %\authorentry{和文著者名}{英文著者名}{所属ラベル}[現在の所属ラベル]\MembershipNumber{}
}
\affiliate[]{}{}
%\affiliate[所属ラベル]{和文所属}{英文所属}
%\paffiliate[]{}
%\paffiliate[現在の所属ラベル]{和文所属}

\begin{document}
\begin{abstract}
  本稿では量子アニーリングを含むイジングモデルを用いたアニーリング法においてスパース推定を可能にするために、近年ReLUタイプ関数のQUBO形式を導出するために用いられたLegendre変換とWolfeの双対定理を利用した。さらに$l1$ノルムに対してこれらの変換を素朴に適用した場合、余分な変数も現れることが明らかになった。最終的に余分な変数を取り除くことで、より簡略化されたQUBO形式を導出する。
\end{abstract}
\begin{keyword}
%和文キーワード 4〜5語
\end{keyword}
\begin{eabstract}
%英文アブストラクト 100 words
\end{eabstract}
\begin{ekeyword}
%英文キーワード
\end{ekeyword}
\maketitle

\section{はじめに}
近年、最適化問題の近似解を得ることに特化したアニーリングマシンが開発・提供されており、代表的なものにカナダのD-Wave.Incの''D-Wave 2000"や富士通の''Digital Annealer"、日立の''CMOSアニーリングマシン"などがある。最適化問題は、データマイニングや機械学習などの様々な分野で利用されている。特に、量子アニーリング法は門脇と西森によって提案され、同様の考え方をもつ断熱量子計算の考え方が注目を集めた。近年では実用化のための研究が行われ、
\section{l1ノルムとスパース推定}
\section{QUBO形式とイジングモデル}
\section{先行研究:$q$-loss関数のQUBO形式での導出}
\subsection{Legendre変換}
\subsection{$q$-loss関数へのLegendre変換の適用}
\section{l1ノルムの素朴なQUBO形式での導出}
\subsection{Wolfeの双対定理}
\section{余分な変数の削除}


\ack %% 謝辞

%\bibliographystyle{sieicej}
%\bibliography{myrefs}
\begin{thebibliography}{99}% 文献数が10未満の時 {9}
\bibitem{}
\end{thebibliography}

\appendix
\section{}

\begin{biography}
\profile*{aaa}{aaa}{aaa}
%\profile{会員種別}{名前}{紹介文}% 顔写真あり
%\profile*{会員種別}{名前}{紹介文}% 顔写真なし
\end{biography}

\end{document}
